\documentclass[11pt]{article}
\usepackage[spanish]{babel}
\usepackage{amsmath,amssymb,amsthm}
\usepackage[utf8]{inputenx}
\usepackage{hyperref}

\title{\huge{\textbf{Análisis asintótico de algoritmos}}}
\author{\large{\textbf{Sergio Revilla Velasco}}}

\date{}

\renewcommand{\P}{\ensuremath{\textup{\textbf{P}}}}

\newcommand{\E}{\ensuremath{\textup{\textbf{E}}}}

\begin{document}
\maketitle
\begin{abstract}
%\noindent
Mediante un desarrollo de la ponencia  \cite{Pon}, demostramos que procesos dinámicos asociados a los números naturales caracterizan al menos un enunciado aritmético con singularidad temporal.
\end{abstract}

\tableofcontents{}
%\setlength{\parindent}{0pt}






\end{document}