\section{Análisis de resultados}

%Gráfica de ordenación por inserción
\subsection{Ordenación por inserci\'on}
Un buen algoritmo de ordenación para un número pequeño elementos. Funciona de la manera análoga a la que se ordenaría un mazo de cartas:
\begin{enumerate}
\item Empezamos con la mano izquierda vacía y las cartas boca abajo sobre la mesa.
\item A continuación, cogemos una carta de la mesa, y la insertamos en la posición correcta en la mano izquierda.
\item Para encontrar la posición correcta en la que insertar la carta, se compara con cada una de las cartas que ya tenemos en la mano izquierda, recorriendo de derecha a izquierda.
\item En todo momento, las cartas sujetas con la mano izquierda se ordenan, permaneciendo las cartas originales sin ordenar encima de la pila en la mesa.
\end{enumerate}

	\begin{figure}[H]
  		\centering
   		 \includegraphics[width=1.0\textwidth]{insertion-sort.png}
  		\caption{Ordenación por inserción}
 			 \label{fig:insertion}
		\end{figure}

%Gráfica de ordenación por selección
\subsection{Ordenación por selecci\'on}
Es un algoritmo de ordenación por comparación. Su complejidad es de $O(n^2)$. Resulta ineficiente en tamaños de problema grandes, y suele funcionar peor que la ordenación por inserción. Se caracteriza por su sencillez, y también tiene ventajas de rendimiento sobre algoritmos más complicados en determinadas situaciones, sobre todo cuando la memoria auxiliar es limitada.
El algoritmo divide la lista de entrada en dos partes: la lista secundaria de los elementos que ya están ordenados, dispuestos de izquierda a derecha, y la lista secundaria de los elementos restantes a ser ordenados, ocupando el resto de la lista. Inicialmente, la sublista ordenada está vacía y la sublista sin ordenar es la lista original menos los elementos ordenados. El algoritmo continúa encontrando siempre el elemento inmediatamente superior (según orden de clasificación) y lo cambia con el último elemento del array.

	\begin{figure}[H]
	  \centering
	    \includegraphics[width=1.0\textwidth]{selection-sort.png}
	  \caption{Ordenación por selección}
	  \label{fig:selection}
	\end{figure}

%Gráfica de ordenación por burbuja
\subsection{Ordenación por burbuja}

	\begin{figure}[H]
	  \centering
	    \includegraphics[width=1.0\textwidth]{bubble-sort.png}
	  \caption{Ordenación por burbuja}
	  \label{fig:bubble}
	\end{figure}

%Gráfica de ordenación rápida
\subsection{Ordenación r\'apida}

	\begin{figure}[H]
  		\centering
   	 	\includegraphics[width=1.0\textwidth]{quick-sort.png}
  		\caption{Ordenación rápida}
  		\label{fig:quick}
	\end{figure}
