\section{Proleg\'omenos}
Todos los algoritmos analizados en este trabajo parten de unas precondiciones y postcondiciones comunes a toda especificación de algoritmo de ordenación, esto es, partiendo de un array de números enteros positivos de tamaño $n > 0$, construimos un nuevo array con los elementos ordenados.  La especificación formal puede expresarse como:
\begin{itemize}
\item \textbf{Precondición:}\\
$P \equiv \{\mbox{lon}(V) = n > 0 \wedge V[n] \geq 0\}$ (Vector no vacío de enteros positivos)
\item \textbf{Definición:}\\$\mbox{ordena}(\mbox{int \:\: \&V} [],\: \mbox{int} \:n)$ (Devuelve el vector ordenado por referencia)
\item \textbf{Postcondición:}\\
$Q \equiv \{\forall{i} : 0 \leq i < n - 1: V[i] \leq V[i + 1]\}$
\end{itemize}
O expresado de otra manera:\cite{CORMEN} 
\begin{itemize}
\item \textbf{Entrada:} una secuencia de $n$ enteros positivos $(a_{1},\: a_{2},\: a_{3}, \:  \ldots\: ,a_{n})$
\item \textbf{Salida:} una permutación (reordenación) de la secuencia de entrada tal que $(a_{1} \leq a_{2}\:  \leq a_{3},\:  \ldots \:  ,\:\leq  a_{n})$
\end{itemize}

Cada método de ordenación se expresa como un algoritmo: un procedimiento de cálculo bien definido que toma algún valor o conjunto de valores como entrada y produce un cierto valor o conjunto de valores como salida. 
